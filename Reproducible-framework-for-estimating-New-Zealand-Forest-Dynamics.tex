% Options for packages loaded elsewhere
\PassOptionsToPackage{unicode}{hyperref}
\PassOptionsToPackage{hyphens}{url}
%
\documentclass[
]{article}
\usepackage{amsmath,amssymb}
\usepackage{lmodern}
\usepackage{iftex}
\ifPDFTeX
  \usepackage[T1]{fontenc}
  \usepackage[utf8]{inputenc}
  \usepackage{textcomp} % provide euro and other symbols
\else % if luatex or xetex
  \usepackage{unicode-math}
  \defaultfontfeatures{Scale=MatchLowercase}
  \defaultfontfeatures[\rmfamily]{Ligatures=TeX,Scale=1}
\fi
% Use upquote if available, for straight quotes in verbatim environments
\IfFileExists{upquote.sty}{\usepackage{upquote}}{}
\IfFileExists{microtype.sty}{% use microtype if available
  \usepackage[]{microtype}
  \UseMicrotypeSet[protrusion]{basicmath} % disable protrusion for tt fonts
}{}
\makeatletter
\@ifundefined{KOMAClassName}{% if non-KOMA class
  \IfFileExists{parskip.sty}{%
    \usepackage{parskip}
  }{% else
    \setlength{\parindent}{0pt}
    \setlength{\parskip}{6pt plus 2pt minus 1pt}}
}{% if KOMA class
  \KOMAoptions{parskip=half}}
\makeatother
\usepackage{xcolor}
\IfFileExists{xurl.sty}{\usepackage{xurl}}{} % add URL line breaks if available
\IfFileExists{bookmark.sty}{\usepackage{bookmark}}{\usepackage{hyperref}}
\hypersetup{
  pdftitle={Controlling invasive predators at a national scale.},
  pdfauthor={Anthony Davidson},
  hidelinks,
  pdfcreator={LaTeX via pandoc}}
\urlstyle{same} % disable monospaced font for URLs
\usepackage[margin=1in]{geometry}
\usepackage{longtable,booktabs,array}
\usepackage{calc} % for calculating minipage widths
% Correct order of tables after \paragraph or \subparagraph
\usepackage{etoolbox}
\makeatletter
\patchcmd\longtable{\par}{\if@noskipsec\mbox{}\fi\par}{}{}
\makeatother
% Allow footnotes in longtable head/foot
\IfFileExists{footnotehyper.sty}{\usepackage{footnotehyper}}{\usepackage{footnote}}
\makesavenoteenv{longtable}
\usepackage{graphicx}
\makeatletter
\def\maxwidth{\ifdim\Gin@nat@width>\linewidth\linewidth\else\Gin@nat@width\fi}
\def\maxheight{\ifdim\Gin@nat@height>\textheight\textheight\else\Gin@nat@height\fi}
\makeatother
% Scale images if necessary, so that they will not overflow the page
% margins by default, and it is still possible to overwrite the defaults
% using explicit options in \includegraphics[width, height, ...]{}
\setkeys{Gin}{width=\maxwidth,height=\maxheight,keepaspectratio}
% Set default figure placement to htbp
\makeatletter
\def\fps@figure{htbp}
\makeatother
\setlength{\emergencystretch}{3em} % prevent overfull lines
\providecommand{\tightlist}{%
  \setlength{\itemsep}{0pt}\setlength{\parskip}{0pt}}
\setcounter{secnumdepth}{5}
\usepackage{booktabs}
\usepackage{amsthm}
\makeatletter
\def\thm@space@setup{%
  \thm@preskip=8pt plus 2pt minus 4pt
  \thm@postskip=\thm@preskip
}
\makeatother
\ifLuaTeX
  \usepackage{selnolig}  % disable illegal ligatures
\fi
\usepackage[]{natbib}
\bibliographystyle{plainnat}

\title{Controlling invasive predators at a national scale.}
\author{Anthony Davidson}
\date{February 2019}

\begin{document}
\maketitle

{
\setcounter{tocdepth}{2}
\tableofcontents
}
Throughout my candatiture I have developed my thesis while simulationously following and attempting to combat the ``Reproducibility crisis'' \citep{peng2015} by writing a series of computationally reproducible pipelines (R packages and github repostories; \ref{repro}) that faciliate the sound curation, analysis and communication of datasets collected for large-scale pest control programs at the national scale \emph{{[}Davidson2020-Reproducibility; Invasive species database{]}} \ref{discus}.

I use this framework to address the ``unexpected'' outcomes of predator removal in two key native forest types in New Zealand \emph{{[}Davidson2020-Beech-forests* \ref{beech}; *MPD-forests{]}} \ref{mpd}. I fit different state-space models to account for the underlying demographic processes of increased complexity. I use New Zealand PFNZ 2050 ``appollo shot'' to demostrate the applicability of this reproducible method to support citizen science and community driven predator control \emph{{[}Davidson2020-PFNZ2050{]}} \ref{discus}.

\hypertarget{general-over}{%
\section{Overview}\label{general-over}}

The overarching aim of this PhD was to examine the role of predation, competition and resource flow in regulating invasive mammal populations in New Zealand forests ecosystems. A key aspect of this research is attempting to account for unexpected outcomes of multi-species predator control. The inherant \ldots.

I have done this by constructing and fitting ecological models that describe the population dynamics of interacting invasive species in New Zealand forests. I use the understanding derived from these models to predict the effects of management manipulations and reduce the likelihood of unanticipated outcomes. Aligning this ecological research with reproducible research practices insures the results presented within this thesis. The final result is a reproducible and repeatable by allowing open access guidelines to document the statistical code and analysis.

This research can be used support other modifications to the forecasting models presented to estimate the likely effects of species removals. In 2016 the New Zealand government funded the ``Predator Free New Zealand 2050''. The disscussion combines the reproducible approach and Bayesian Modelling techniques to input additional obsersvational data to test the many othr ecosystem models that allow researchers to directly quantify the interactions among invasive species \citep{Peng2015}.

\hypertarget{this-thesis-generates}{%
\subsection{This thesis generates:}\label{this-thesis-generates}}

\begin{enumerate}
\def\labelenumi{\arabic{enumi}.}
\item
  A reproducible and scalable research database for invasive mammals in NZ.
\item
  Advanced in ecological models that incorperate both the theoretical models and observational data \citep{King2012}.
\item
  Interactive tools to develop understanding and application of Bayesian Hierarichal Models for conservation managers and citizen researchers alike.
\end{enumerate}

  \bibliography{book.bib,packages.bib,phd-used-references.bib}

\end{document}
